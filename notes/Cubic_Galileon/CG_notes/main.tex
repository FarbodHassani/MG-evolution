\documentclass{article}

\usepackage{xcolor}
\usepackage{amsmath}
\usepackage{amssymb}
\usepackage{url}
\usepackage{hyperref}
\usepackage{todonotes} % Add to-do notes


\newcommand{\fh}[1]{\textcolor{blue}{#1}}
\newcommand{\fhc}[1]{\textcolor{red}{[#1]}}
\def\be{\begin{equation}}
\def\ee{\end{equation}}


\title{The Galileon model}
\author{Farbod Hassani}
\date{\today} 

\begin{document}
\maketitle 

\section{The Galileon model}
In Galileon model  the modifications to GR arise through a
Galilean-invariant scalar field, i.e., the scalar field is invariant under the
transformation $\partial_{\mu} \phi \to \partial_{\mu} \phi + b_{\mu}$, where $b_{\mu}$ is a constant vector.

The action of the cubic Galileon model, for when the scalar field does not have a direct coupling matter sector reads as,
\be
S = \int d^4x \sqrt{-g} \left[ \frac{R}{16\pi G} - \frac{1}{2}c_2\mathcal{L}_2 - \frac{1}{2}c_3\mathcal{L}_3 - \mathcal{L}_m \right],
\ee
where $g$ is the determinant of the metric $g_{\mu \nu}$, $R$ is the Ricci scalar, $c_2, c_3$ are dimensionless constants and $\mathcal{L}_m$ is the matter Lagrangian. The terms $\mathcal{L}_2$ and $\mathcal{L}_3$ in the Lagrangian are describing the Galileon field, and $\nabla$ denotes the covariant derivative.
\be
\mathcal{L}_2 = \nabla_\mu \phi \nabla^\mu \phi, \quad \mathcal{L}_3 = \frac{2}{M^3} \Box \phi \nabla_\mu \phi \nabla^\mu \phi,
\ee
where $\phi$ is the Galileon scalar field and $\mathcal{M}^3 \equiv M_{\mathrm{Pl}} H_0^2$ where $H_0$ is the value Hubble function today.
The Einstein field equations read:
\todo{write some details}
\begin{equation}
G_{\mu\nu} = \kappa \left[ T_{\mu\nu}^m + T_{\mu\nu}^{c_2} + T_{\mu\nu}^{c_3} \right],
\end{equation}
\todo{derive!} and the scalar field equation,
\begin{equation}
c_2 \Box \phi + \frac{2}{M^3} c_3 \left[ (\Box \phi)^2 - \nabla^\alpha \nabla_\beta \phi \nabla^\beta \nabla_\alpha \phi \right] - R_{\alpha\beta} \nabla^\alpha \phi \nabla^\beta \phi = 0,
\end{equation}
where the energy-momentum tensor reads as \todo{derive the equations by hand and using mathematica!}
\begin{equation}
T_{\mu\nu}^{c_2} = c_2 \left[ \nabla_\mu \phi \nabla_\nu \phi - \frac{1}{2} g_{\mu\nu} \nabla^\alpha \phi \nabla_\alpha \phi \right],
\end{equation}
\begin{equation}
T_{\mu\nu}^{c_3} = \frac{c_3}{M^3} \left[ 2 \nabla_\mu \phi \nabla_\nu \phi \Box \phi + 2 g_{\mu\nu} \nabla^\alpha \nabla^\beta \phi \nabla_\alpha \nabla_\beta \phi - 4 \nabla^\lambda \phi \nabla_{(\mu} \phi \nabla_{\nu)} \nabla_\lambda \phi \right],
\end{equation}
and $T^m_{\mu \nu}$ is the energy-momentum tensor of other species in the universe.

It is worth noting that the general Galileon action has two more Lagrangian densities $\mathcal{L}_4$ and  $\mathcal{L}_5$, which are quartic and quintic in derivatives of scalar field. However, the higher order derivatives make the theory suffer from various theoretical problems. There is also a linear term in the general Galileon theory $\mathcal{L}_1$ which is often neglected if the scalar field plays the role of dark energy.
\subsection{Background evolution}
To derive the background equations we assume a FLRW metric,
\be
ds^2 = dt^2 - a(t)^2 \delta_{ij} dx^i dx^j
\ee
then the Friedmann equations read,
\begin{equation}
3H^2 = 8 \pi G \left[ \rho_m + \frac{c_2}{2} \dot{\bar \phi}^2 + \frac{6 c_3}{M^3}\dot{ \bar\phi}^3 H \right],
\end{equation}
\begin{equation}
\dot{H} + H^2 = -\frac{4 \pi G}{3} \left[ \rho_m + 4c_2\dot{\bar \phi}^2 \right] + \frac{6 c_3}{M^3} \left( \dot{\bar \phi}^3 H - \dot{\bar \phi}\ddot{ \bar\phi} \right),
\end{equation}
where the derivatives and quantities are calculated with respect to physical time, $\rho_m$ is the matter density and we have neglected the contribution from the radiation. The Galileon field equation of motion reads,
\begin{equation}
 c_{2}[\ddot{\bar \phi}+3 \dot{\bar\phi} H] 
 +6 \frac{c_{3}}{\mathcal{M}^{3}}\left[2 \ddot{\bar\phi} \dot{\bar\phi} H+3 \dot{\bar\phi}^{2} H^{2}+\dot{\bar\phi}^{2} \dot{H}\right] = 0
\end{equation}
\subsection{Perturbation, quasi-static approximation}
The derive the perturbation equations we assume a perturbed FLRW metric in Newtonian gauge,
\be
d s^{2}=(1+2 \Psi) d t^{2}-a(t)^{2}(1-2 \Phi) \delta_{i j} d x^{i} d x^{j},
\ee
and we assume that the gravitational potentials are small and $\phi(\vec x, t) = \bar{\phi}(t) + \delta \phi(\vec x, t)$. Moreover we assume the quasi-static limit, which seems to be a good approximation in the sub-horizon scales. In the quasi-static limit we neglect the time derivative of scalar field compared with the spatial derivatives. Moreover additionally, the Newtonian potentials $\Phi, \Psi$ and their first order spatial derivatives $\Phi_{,i}$ and $\Psi_{,i}$ can be neglected compared with the $\nabla^2 \Phi$ as the Newtonian potential and their first order derivative are small in the scales of interest. So we have $(1+ 2 \Psi) \partial_i \partial^i \phi \approx \partial_i \partial^i \phi$ and \(\partial_{i} \partial^{i} \Phi \partial_{j} \Phi \ll \partial_{i} \partial^{i} \Phi\). Under the previously mentioned approximations the (0,0) component of the Einstein field equations (the Poisson equation) reads
\be
\partial^2 \Phi=4 \pi G a^2 \delta \rho_m-\frac{\kappa c_3}{\mathcal{M}^3} \dot{\phi}^2 \partial^2 \phi.
\ee
The scalar field equations,
\be
\frac{2 c_3}{\mathcal{M}^3} \dot{\phi}^2 \partial^2 \Psi=\left[-c_2-\frac{4 c_3}{\mathcal{M}^3}(\ddot{\phi}+2 H \dot{\phi})\right] \partial^2 \phi
+\frac{2 c_3}{a^2 \mathcal{M}^3}\left[\left(\partial^2 \phi\right)^2-\left(\partial_i \partial_j \phi\right)^2\right] .
\ee
Where we have not used $\delta \phi$ and we write it as $\phi$ but from the order of equations we can guess whether $\bar \phi$ or $\delta \phi$ should be used. \\
Assuming $\Phi = \Psi$ we can combine the two equations,
\be
\partial^{2} \phi+\frac{1}{3 \beta_{1} a^{2} \mathcal{M}^{3}}\left[\left(\partial^{2} \phi\right)^{2}-\left(\partial_{i} \partial_{j} \phi\right)^{2}\right]
=\frac{M_{\mathrm{Pl}}}{3 \beta_{2}} 8 \pi G a^{2} \delta \rho_{m}
\ee
Here, \(\partial^{2}=\partial_{i} \partial^{i}\) is the spatial Laplacian, \(\left(\partial_{i} \partial_{j} \phi\right)^{2}=\left(\partial_{i} \partial_{j} \phi\right)\left(\partial^{i} \partial^{j} \phi\right) . \delta \rho_{m}\) is the matter density perturbation, \(\rho_{m}=\bar{\rho}_{m}(t)+\delta \rho_{m}(t, \vec{x})\). The dimensionless functions \(\beta_{1}\) and \(\beta_{2}\) are defined as

\[
\begin{aligned}
& \beta_{1}=\frac{1}{6 c_{3}}\left[-c_{2}-\frac{4 c_{3}}{\mathcal{M}^{3}}(\ddot{\phi}+2 H \dot{\phi})+2 \frac{\kappa c_{3}^{2}}{\mathcal{M}^{6}} \dot{\phi}^{4}\right] \\
& \beta_{2}=2 \frac{\mathcal{M}^{3} M_{\mathrm{Pl}}}{\dot{\phi}^{2}} \beta_{1} .
\end{aligned}
\]
\subsection{Vainshtein screening}
In order to observe the screening through the derivative couplings, it is instructive to use spherical symmetry. Where we can rewrite the equation as,
\be
\frac{1}{r^{2}} \frac{d}{d r}\left[r^{2} \phi_{,r} \right]+\frac{2}{3} \frac{1}{\mathcal{M}^{3} a^{2}\beta_{1} } \frac{1}{r^{2}} \frac{d}{d r}\left[r \phi,{ }_{r}^{2}\right]  =\frac{M_{\mathrm{Pl}}}{3 \beta_2} 8 \pi G a^{2} \delta \rho
\ee
Integrating it once,
\be
\phi,_{r}+\frac{2}{3} \frac{1}{\mathcal{M}^{3} a^{2}\beta_{1} } \frac{1}{r} \phi,_{r}^{2}=\frac{2 M_{\mathrm{Pl}}}{3 \beta_2} \frac{G M(r)}{r^{2}} a^{2}
\ee
Now the equation becomes an algebraic equation for $\phi_{,r}$ which considering a mass profile can be solved. And \(M(R)=4 \pi \int_{0}^{R} \delta \rho_{m}(r) r^{2} d r\) is the matter contribution to the mass enclosed within a radius \(R\). For simplicity considering a top-hat density distribution of radius R, the physical solutions of $\phi_{,r}$ are,
\be
\phi_{,r}=\frac{4 M_{\mathrm{Pl}} a^{2} r^{3}}{3 \beta r_{V}^{3}}\left[\sqrt{\left(\frac{r_{V}}{r}\right)^{3}+1}-1\right] \frac{G M(R)}{r^{2}}, \,\,\, r \geq R
\ee
\be
\phi_{,r}=\frac{4 M_{\mathrm{Pl}} a^{2} R^{3}}{3 \beta r_{V}^{3}}\left[\sqrt{\left(\frac{r_{V}}{R}\right)^{3}+1}-1\right] \frac{G M(r)}{r^{2}}, \,\, r<R
\ee
Where we have used the field redefinition \[
\delta \phi \rightarrow \frac{\beta}{\beta_{2}} \delta \phi
\] as discussed in section 2 of  \href{https://arxiv.org/abs/1306.3219}{1306.3219}. 
The distance scale $r_V$ is the Vainshtein radius and is given by,
\be
r_{V}^{3}=\frac{8 M_{\mathrm{P}l} r_{S}}{9 \mathcal{M}^{3} \beta_{1} \beta_{2}},
\ee
where \(r_{S} \equiv 2 G M(R)\)  is the Schwarzschild radius of the top-hat mass density.\\

Considering the last term in the modified Poisson equation, it represents a fifth force due to the presence of Galileon scalar field.
\be
F_{5 t h}=-\frac{\kappa c_{3}}{\mathcal{M}^{3}} \frac{\beta}{\beta_{2}} \dot{\phi}^{2} \phi_{,r}
\ee
Taking the limits where \(r \gg r_{V}\) and \(r \ll r_{V}\) we haves
\be
F_{5 t h}  =-\frac{2 c_{3} a^{2} \dot{\phi}^{2}}{3 \mathcal{M}^{3} M_{\mathrm{Pl}} \beta_{2}} \frac{G M(R)}{r^{2}},  \; \; \; r \gg r_{V}, 
\ee
\be
F_{5 t h}  \sim 0,  \; \; \; r \ll r_{V} .
\ee
\section{MG-evolution implementation}
In this section we discuss the approximations and framework we use to obtain the expression for implementing in MG-evolution.
\subsection{Linear part}
The effective gravitational potential in Cubic Galileon theory at linear regime reads as (Eq. 29 of \href{https://arxiv.org/abs/1306.3219}{1306.3219})
\be
G_{\mathrm{eff}}=G\left(1-\frac{2}{3} \frac{c_3 \dot{\phi}^2}{M_{\mathrm{Pl}} \mathcal{M}^3 \beta_2}\right)
\ee
where $M_{\mathrm{Pl}}$ is the Planck mass, $\phi$ is the Galileon scalar field and $\mathcal{M}^3 \equiv M_{\mathrm{Pl}} H_0^2$, $\beta_1$ and $\beta_2$ are,
\begin{align*}
\beta_1 &= \frac{1}{6 c_3}\left[-c_2 - \frac{4 c_3}{\mathcal{M}^3}(\ddot{\phi}+2 H \dot{\phi}) + 2 \frac{\kappa c_3^2}{\mathcal{M}^6} \dot{\phi}^4\right], \\
\beta_2 &= 2 \frac{\mathcal{M}^3 M_{\mathrm{Pl}}}{\dot{\phi}^2} \beta_1. \\
\kappa &= \frac{1}{M_{\mathrm{Pl}}^2} = 8 \pi G
\end{align*}
In our discussion we set $c_2 = -1$ (see discussion at page 5 of \href{https://arxiv.org/pdf/1709.09135.pdf}{1709.09135}). We also use the tracker solution,
\be
\xi \equiv \frac{\dot{\phi} H}{M_{\mathrm{Pl}} H_0^2}
\ee
where $\xi$ is a constant. It is important to note that in cases where $c_2/c_3^{2/3}$ and $c_3$ is given similar to the first table in \href{https://arxiv.org/abs/1306.3219}{1306.3219} we need to rescale $c_2$ to $-1$. So for example if,
\be
 c_2/c_3^{3/2} = -5.378 , c_3 = 10
\ee
which is equivalent to 
\be
 c_2= -24.9624 , c_3 = 10
\ee
If we convert these to our notation where $c_2 = -1$, then we obtain 
\be
c_3 = 10/(24.9624)^{3/2} = 0.080, \xi = -\frac{c_2}{6 c_3 } = 2.0786
\ee
where it follows from the fact that to get $c_2 = -1$ and since we have $\mathcal{L}_2 \sim \nabla_{\mu} \phi \nabla^{\mu} \phi$. We need to redefine the field as
\be
 \phi \to \phi/(-c_2)^{1/2}
 \ee
 As $\mathcal{L}_3 \sim \phi^3$ ignoring the derivatives, we get
 \be 
 c_3 \to c_3/(-c_2)^{3/2}
 \ee

Following the discussion which leads to E. 18 in \href{https://arxiv.org/pdf/1709.09135.pdf} {1709.09135} we can deduce that $\xi$ is a constant and can be obtained given $c_3$,
\be
\xi = -\frac{1}{6 c_3}
\ee
As a result we can write,
\be
\dot{\phi} =   \frac{\xi M_{\mathrm{Pl}} H_0^2}{H} 
\ee
\be
\ddot{\phi} =  - \frac{\xi M_{\mathrm{Pl}} H_0^2 \dot{H}}{H^2} 
\ee
Note that $H = \frac{d a}{dt} = \mathcal{H}/a$, where $\mathcal{H}$ is the conformal Hubble factor. Following the discussion presented in \href{https://arxiv.org/pdf/1308.3699.pdf}{1308.3699} for the background tracker solution we can derive the Hubble expansion rate as a function of $a$ (Eq. 12 of \href{https://arxiv.org/pdf/1308.3699.pdf}{1308.3699})
\begin{equation}
\begin{aligned}
\mathcal{H}^2 &= \frac{\mathcal{H}_0^2}{2}\left[\left(\Omega_{m0} a^{-1} + \Omega_{r0} a^{-2}\right) + a^2 \sqrt{\left(\Omega_{m0} a^{-3} + \Omega_{r0} a^{-4}\right)^2 + 4\left(1 - \Omega_{m0} - \Omega_{r0}\right)}\right].
\end{aligned}
\end{equation}
Where $H_0^2 = \frac{8 \pi G}{3}$ in MG-evolution unit.Computing $\mathcal{H}'$ results in,
\begin{equation}
\begin{aligned}
\mathcal{H}' = & -\frac{\mathcal{H}_0^2 (a \Omega_m + 2 \Omega_r)}{4 a^2} 
 - \frac{\mathcal{H}_0^2(a \Omega_m + \Omega_r) (3 a \Omega_m + 4 \Omega_r)}{4 a^6 \sqrt{4 (1 - \Omega_m - \Omega_r) + \frac{(a \Omega_m + \Omega_r)^2}{a^8}}} \\
& + \frac{\mathcal{H}_0^2 a^2}{2} \sqrt{4 (1 - \Omega_m - \Omega_r) + \frac{(a \Omega_m + \Omega_r)^2}{a^8}}
\end{aligned}
\end{equation}
We also have,
\be
\dot{H} = \frac{\mathcal{H}' - \mathcal{H}^2}{a^2}
\ee
With all the previous expressions we can obtain the linear $\Delta G/G$ for the Cubic Galileon model. 
\section{Screening part}
Following the discussion which leads to Eq. 21 of \href{https://arxiv.org/abs/1306.3219}{1306.3219} we can write,
\be
\frac{\Delta G}{G}|_{\rm tot} =  \frac{\Delta G}{G}|_{\rm linear}  \times \frac{\Delta G}{G}|_{\rm Vainshtein}
\ee
where $\frac{\Delta G}{G}|_{\rm linear}$  is obtained following the discussion in the previous section and  $\frac{\Delta G}{G}|_{\rm Vainshtein}$ can be written in Fourier space as \href{https://arxiv.org/pdf/2003.05927.pdf}{2003.05927},
\be
\frac{\Delta G}{G}|_{\rm Vainshtein} = \frac{\sqrt{1+ \epsilon}-1}{\epsilon}
\ee
where,
 \be
 \epsilon \equiv (\frac{r_V}{r})^3 \to (\frac{k_*}{k})^3
 \ee
where $r_V$ the Vainshtein radius and $k_*$ is the corresponding wavenumber in Fourier space. As a result we have two free parameters, namely $k_*$ in the screening and $c_3$ in the linear modification.


\end{document}
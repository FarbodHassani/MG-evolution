

\documentclass[a4paper,10pt]{article}
\pdfoutput=1
\usepackage{jcappub}
\usepackage{bbold}

\usepackage[english]{babel}
\usepackage[utf8]{inputenc}
\usepackage{amsmath}
\usepackage{color}
\usepackage{amsfonts}
\usepackage{graphicx}
\usepackage{amssymb}
\usepackage{eufrak}
\usepackage{etoolbox}
\usepackage{amsmath}
\usepackage{empheq}
\usepackage{cancel}
\usepackage[most]{tcolorbox}
\usepackage{float}              % Activate [H] option to place figure HERE

\newtcbox{\mymath}[1][]{%
    nobeforeafter, math upper, tcbox raise base,
    enhanced, colframe=yellow!30!black,
    colback=yellow!30, boxrule=1pt,
    #1}
%%%%%%%%%%%%%%%%%%%%%%%%%%%%

\def\be{\begin{equation}}
\def\ee{\end{equation}}
\def\bea{\begin{eqnarray}}
\def\eea{\end{eqnarray}}
\def\bean{\begin{eqnarray*}}
\def\eean{\end{eqnarray*}}
\def\cd{\cdot}
\def\vp{\varphi}
\def\l {\langle}
\def\re {\rangle}
\def \dd {\partial}
\def \ra {\rightarrow}
\def \la {\lambda}
\def \La {\Lambda}
\def \De {\Delta}
\def \DH {\Delta_{\rm HI}}
\newcommand{\de}{\delta}
\def \b {\beta}
\def \al {\alpha}
\def \ka {\kappa}
\def \Ga {\Gamma}
\def \ga {\gamma}
\def \si {\sigma}
\def \Si {\Sigma}
\def \ep {\epsilon}
\def \om {\omega}
\def \Om {\Omega}
\def \lap {\triangle}
\def \ep {\epsilon}


%%%%%%%%%%%%%%%%%%%%%%%%%%%%%%%%%%%
%Special definitions for this paper
%%%%%%%%%%%%%%%%%%%%%%%%%%%%%%%%%%%

\newcommand{\MyRed}{\color [rgb]{0.8,0,0}}
\newcommand{\MyGreen}{\color [rgb]{0,0.7,0}}
\newcommand{\MyBlue}{\color [rgb]{0,0,0.8}}
\newcommand{\MyBrown}{\color [rgb]{0.8,0.4,0.1}}
\newcommand{\MyPurple}{\color [rgb]{0.6,0.0,0.6}}
\def\GV#1{{\MyRed [GV: #1]}}
\def\RD#1{{\MyGreen [RD:  {\tt #1}]}} 
\def\RDt#1{{\MyGreen #1}}   
\def\GM#1{{\MyBlue [GM: #1]}}  
\def\GF#1{{\MyPurple [GF: #1]}}    



\newcommand{\ie}{\emph{i. e.}}
\newcommand{\cf}{\emph{cf.}}
\newcommand{\etal}{\emph{et al.}\xspace}
\newcommand{\eg}{\emph{e. g.}}

\newcommand{\Scal}{\mathcal S}
\newcommand{\DD}{\mathcal D}
\newcommand{\EE}{\mathcal E}
\newcommand{\MM}{\mathcal M}
\newcommand{\HH}{\mathcal H}

\newcommand{\Real}{\mathbb{R}}
\newcommand{\bn}{\boldsymbol{n}}
\newcommand{\bv}{\boldsymbol{v}}
\newcommand{\bx}{\boldsymbol{x}}
\newcommand{\bnabla}{\boldsymbol{\nabla}}
\newcommand{\bell}{\boldsymbol{\ell}}
\newcommand{\bal}{\boldsymbol{\alpha}}


%Farbod commands
\newcommand{\Ge}{G_{\text{eff}}}
\newcommand{\MP}{M_{\text{pl}}}

%%%%%%%%%%%%%%%%%%%%%%%%%%%%%%%%%%%%%%%%%%



\title{... }

\author[a]{.}
\author[b]{, .}
\author[c]{, .}
\author[d]{,.}
\author[e]{,.}

%\affiliation[a]{
%Universit\'e de Gen\`eve, D\'epartement de Physique Th\'eorique and CAP,
%24 quai Ernest-Ansermet, CH-1211 Gen\`eve 4, Switzerland
%}

%\emailAdd{farbod.hassani@unige.ch}
%\emailAdd{martin.kunz@unige.ch}
\emailAdd{..}
\emailAdd{..}

\abstract{
In order to verify GR with the upcoming surveys we need to compare the cosmological quantities inferred by General Relativity (GR) assumptions  with the ones with non-GR  assumptions. In other words we need to compute the cosmological quantities (like $f \sigma_8$ ) consistently in each theory of gravity, including the specific background and perturbations. On the other hand, the effect of different dark energy models on linear and non-linear scales is still unknown and is not studied consistently. We are going to modify the Newtonian part of N-body code "Gevolution" to implement the  effect of modified gravity models like DGP on scalar perturbations. To parametrize the possible deviations from GR, we use two parameters $\mu(k,z)$ and $\eta(k,z)$ and we will discuss about the validity of GR by probing the parametric space of $\gamma$ and $\mu$...
}

\begin{document}
\maketitle
%%%%%%%%%%%%%%%%%
%%%INTRODUCTION
%%%%%%%%%%%%%%%%%
\section{Introduction}
\section{Modification of Gravity: DGP }
\subsection{Background}
The Friedman equation for $k=0$ in DGP model reads as following (eq.20 of \url{https://arxiv.org/pdf/astro-ph/0105068.pdf} )
\be
\mathcal{H}^2 (z) = (1+z)^{-2} \HH_0^2 \Bigg(\sqrt{\Omega_{rc}} + \sqrt{\Omega_{rc} + \sum_{\alpha} \Omega_{\alpha} (1+z)^{3 (1+w_{\alpha})} }  \Bigg)
\ee
where in Gevolution $\HH_0^2 = \HH = \frac{8\pi G}{3}$. To obtain $\Omega_{rc}$ we need to solve an algebraic equation at $z=0$
\be
 \sqrt{\Omega_{rc}} + \sqrt{\Omega_{rc} + \sum_{\alpha} \Omega_{\alpha} }  =1
\ee
To solve the equation we use Newton's method. We want to solve the equation
 \be
 f(\Omega_{rc})= \sqrt{\Omega_{rc}} + \sqrt{\Omega_{rc} + \sum_{\alpha} \Omega_{\alpha} }   -1 =0 
 \ee
 We start from a guess like $\Omega_{rc} \to 0.11289$ which is a value for LCDM background. Then we compute $f(\Omega_{rc} =0.11289 )$ and $f'(\Omega_{rc} =0.11289 )$ to find the next solution as and we continue until the two successive solution converge with a certain precision.
 \be
 x_{n+1} = x_n - \frac{f(x_n)}{f'(x_n)}
 \ee

\subsection{Perturbations}
We use the Newtonian flag of Gevolution  and modify the Poisson equation as following
 \be
\nabla \Psi_N = 4 \pi G{a^2} \Big(1+ \frac{\Delta G_{\text{eff}}}{G} \Big)\; \delta \rho
  \ee
  For the Vainshtein screening of DGP we have (\url{https://arxiv.org/pdf/1608.00522.pdf})
  
\be
\frac{\Delta G_{\text{eff}}}{G} = 2 \Big( \frac{r}{r_0} \Big)^3 \Big[ \sqrt{1+ ( \frac{r_0}{r} )^3} -1 \Big]
\ee
we take $(\frac{r_0}{r})^3 = \frac{\rho_m} {\bar{\rho}} = 1+\delta_m  $. 
%%%%%%%%%%%%%%%%%%%%%%%%%%%%%%%%%%%%%%%%%%%%%
We basically need to modify the right hand side of the equation in the Gevolution and then Fourier transform it to solve the modified Poisson equation! In the Gevolution the matter stress tensor $T_0^0$ is made as following,
\be
T_0^0 (\text{Gev}) = -a^3 T_0^0 = \bar{\rho}_m a^3 \big[1 + \delta_m \big] =  \bar{\rho}^{(0)}_m \big[1 + \delta_m \big]
\ee
So in the code to have $(\frac{r_0}{r})^3$, we can write
\be
(\frac{r_0}{r})^3 = 1+\delta_m  = \frac{T_0^0 (\text{Gev}) }{\bar{\rho}^{(0)}_m}
\ee
I assume "source" field in Gevolution for Newtonian gravity is $\delta \rho$ so in order to make $\delta_m$ from it we need to have $\bar 
\rho_m$ which is,
\be
\bar{\rho}_m = \bar{\rho}_m^0 a^{-3}
\ee
In sum we have,
\be
\frac{\Delta G_{\text{eff}}}{G} = \frac2 {1+ \delta_m} \Big[ \sqrt{2+ \delta_m} -1 \Big]
\ee
-Questions:\\
What is the initial redshift, we can start simulation? are the equations self-consistent that we do not modify gravity much in high redshifts?

\section{Results}
\subsection{Background}

%%%%%%%%%%%%%%%%%%%%%%%%%%%%%
\section{Conclusions}

\setcounter{equation}{0}
%%%%%%%%%%%%%%%%%%%%%

 
\section*{Acknowledgements}

We acknowledge financial support from the Swiss National Science Foundation.


%%%%%%%%%%%%%%%%%%%%%%%%%%%%%%%
%%%%%%%%%%%%%%%%%%%%%%%%%%%%%%%
\appendix
%
\bibliographystyle{JHEP}
%\bibliography{biblio}
\bibliography{EFTDE}


\begin{thebibliography}{999}
\newcommand{\bb}{\bibitem}
%\cite{Hojjati:2011ix}
\bibitem{Hojjati:2011ix}
  A.~Hojjati, L.~Pogosian and G.~B.~Zhao,
  %``Testing gravity with CAMB and CosmoMC,''
  JCAP {\bf 1108} (2011) 005
  doi:10.1088/1475-7516/2011/08/005
  [arXiv:1106.4543 [astro-ph.CO]].
  %%CITATION = doi:10.1088/1475-7516/2011/08/005;%%
  %125 citations counted in INSPIRE as of 20 Sep 2018
 \end{thebibliography}


\end{document}


